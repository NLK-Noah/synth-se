\documentclass{article}
\usepackage{graphicx}
\usepackage[french]{babel}
\usepackage{multicol}
\usepackage{geometry}
\usepackage{titling}
\usepackage[utf8]{inputenc}
\usepackage{natbib}
\usepackage[style=authoryear,backend=biber]{biblatex}
\usepackage{hyperref}

\addbibresource{name.bib}




\geometry{
    a4paper,
    total={170mm,257mm},
    left=20mm,
    top=30mm,
}



\title{
    \includegraphics[width=1\textwidth]{photo/UCLouvain_Charleroi.png} \\
    \vspace{1.5cm}
    {\Huge \textbf{Rapport personnelle projet vis}} \\
    \vspace{1.5cm}
}

\author{
    \textbf{Moussaoui Noah} \\
    Université catholique de Louvain-la-Neuve \\
    Campus de Charleroi, EPL en SINC \\
    Délégué et ambassadeur \\
    \texttt{Noah.moussaoui@student.uclouvain.be}
}

\date{
    \vspace{1.5cm}
    Durée de recherche : Septembre - Novembre \\
     \vspace{1.5cm}
    \includegraphics[width=0.5\textwidth]{photo/EPL.jpeg}
}

\begin{document}

\maketitle
\newpage
\section{Introduction}
Dans ce projet, nous avons travaillé sur une application mobile dont l'objectif est de permettre 
aux utilisateurs de rencontrer des requins, mais sans adopter le format traditionnel des applications
de rencontres. Il s'agit plutôt d'un réseau social où les utilisateurs peuvent se retrouver pour 
"chasser" des humains, consulter des publications, découvrir des propositions d’événements, 
et regarder des stories.\\\\

Ensemble, nous nous appuyons sur les données d’un fichier CSV contenant des informations 
sur différentes attaques de requins dans le monde, données qui alimenteront bien sûr notre application.


\section{Personas}


\section{Base-fidélité}


\section{conclusion}


\end{document}
